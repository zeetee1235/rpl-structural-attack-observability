\documentclass[11pt,a4paper]{article}
\usepackage{kotex}
\usepackage[margin=1in]{geometry}
\usepackage{graphicx}
\usepackage{booktabs}
\usepackage{array}
\usepackage{longtable}
\usepackage{amsmath,amssymb}
\usepackage{float}
\usepackage{caption}
\usepackage{hyperref}

\title{RPL 기반 IoT 네트워크에서 토폴로지 및 라우팅 구조에 따른\\Selective Forwarding 공격 관측 가능성 분석\\\large 연구계획서}
\author{작성자: 이건형}
\date{\today}

\begin{document}
\maketitle

\begin{abstract}
본 연구는 RPL/BRPL 기반 IoT 네트워크에서 Selective Forwarding 공격이
\emph{언제 성능 지표로 관측 가능한지}를 분석한다.
공격 탐지 기법을 제안하기보다, 토폴로지 및 parent 선택 구조를 나타내는 구조적 지표(평균 경로 길이, 경로 다양성, 중심성, 공격 경유 확률)와
관측 지표(PDR 저하, 지연 증가)의 관계를 정량화하는 데 목적이 있다.
선행 BRPL 실험 로그 분석 결과, 동일한 공격률(50\%)에서도 토폴로지별 관측 가능성 편차가 크다는 점이 관찰되었으며,
이는 ``공격률이 같아도 구조에 따라 관측 가능성이 크게 달라진다''는 본 연구의 동기를 제공한다.
\end{abstract}

\section{서론}
시뮬레이션 및 테스트베드 실험에서 Selective Forwarding 공격률이 증가해도 PDR/Delay 변화가 미미한 사례가 반복 관찰되었다.
이는 ``공격률 증가 $\Rightarrow$ 성능 저하''라는 단순 가정이 항상 성립하지 않음을 보여준다.
특히 RPL/BRPL과 같이 parent가 시간에 따라 바뀌는 동적 라우팅 환경에서는,
공격이 존재하더라도 경로 재선택으로 인해 단기 성능 지표에서는 변화가 희석되거나 지연 관측될 수 있다.

본 연구는 다음과 같은 질문에 답하고자 한다.
\begin{itemize}
  \item 어떤 네트워크 구조에서 Selective Forwarding 공격이 명확히 관측되는가?
  \item RPL/BRPL parent 선택 구조가 공격 관측 가능성에 어떤 영향을 주는가?
  \item 공격 노드의 위치(깊이/중심성)는 관측 가능 조건을 어떻게 바꾸는가?
\end{itemize}

본 연구는 공격 \emph{탐지 기법}을 제안하지 않고,
공격 효과가 PDR/지연과 같은 성능 지표로 \emph{드러나는 조건(관측 가능 조건)}을 구조적으로 정량화하는 데 초점을 둔다.

\begin{figure}[H]
  \centering
  % R 기반으로 다시 작성한 연구 흐름 개요도 (PDF 벡터 그림 사용)
  \includegraphics[width=0.98\linewidth]{figures/fig_research_flow}
  \caption{연구 흐름 개요: 연구 질문 $\Rightarrow$ 구조 지표 $\Rightarrow$ 시뮬레이션 $\Rightarrow$ 관측 지표 $\Rightarrow$ 분석}
  \label{fig:research-flow}
\end{figure}

\noindent
본 절에서는 Selective Forwarding 공격의 문제의식과 본 연구의 위치를 개괄적으로 제시하였다.

\section{핵심 관점 및 구조적 지표 개념}
무선 RPL 환경은 parent가 시간에 따라 변하므로 고정 토폴로지(스타/체인/메쉬 등)를 가정하기 어렵다.
본 연구는 토폴로지를 소수의 라벨로 분류하기보다,\textbf{구조적 지표 벡터}로 요약하여 분석하는 관점을 취한다.

\subsection{구조적 지표 정의}
\paragraph{(1) 공격 노드 경유 확률}
\[
P_{\text{attack}}(i) = \sum_{p \in \mathcal{P}_i} P(p)\,\mathbb{I}(A \in p)
\]
여기서 $\mathcal{P}_i$는 노드 $i$에서 루트까지의 가능한 경로 집합이다.

\paragraph{(2) 평균 경로 길이 (APL)}
\[
APL = \frac{1}{N}\sum_i \mathbb{E}[\text{hop count}_i]
\]

\paragraph{(3) 경로 다양성 (PD)}
\[
PD(i) = |\mathcal{P}_i|
\]
실험에서는 관측 구간 내 서로 다른 parent 수로 근사한다.

\paragraph{(4) 공격 노드 중심성 (Betweenness Centrality)}
\[
BC(A)=\sum_{s\neq t}\frac{\sigma_{st}(A)}{\sigma_{st}}
\]

\subsection{관측 모델(초안)}
공격률을 $\alpha$라 할 때 노드 $i$의 기대 PDR은 다음과 같이 근사한다.
\[
\mathbb{E}[PDR_i] \approx 1 - \alpha\,P_{\text{attack}}(i)
\]
또한 통계적 관측 가능 조건을 다음과 같이 둔다.
\[
\alpha\,P_{\text{attack}}(i) > \varepsilon(\sigma)
\]
여기서 $\varepsilon(\sigma)$는 링크 품질 변동, 트래픽 변동, MAC 충돌 등 노이즈를 포함한다.

\begin{figure}[H]
  \centering
  % 구조 지표 개념 스케치 (PDF 벡터 그림 사용)
  \includegraphics[width=0.9\linewidth]{figures/fig_structural_metrics}
  \caption{예시 RPL 트리 위에서 APL, 경로 다양성, 공격 노드 중심성 개념 스케치}
  \label{fig:struct-metrics}
\end{figure}

\noindent
본 절에서는 공격 경유 확률, 평균 경로 길이, 경로 다양성, 중심성과 같은 구조적 지표와 관측 모델(초안)을 정의하였다.

\section{선행 실험 중 우연히 관찰된 패턴}
본 섹션의 결과는 Trust-Aware BRPL 관련 선행 실험(\url{https://github.com/zeetee1235/trust-aware-brpl})을 수행하는 과정에서 관찰된 현상을 정리한 것이다.
체계적으로 설계된 ``예비 실험"이라기보다는, 다른 목적의 실험 로그를 분석하던 중 발견한 패턴을 현재 연구의 문제 인식에 활용한 사례로 본다.

\subsection{Topology별 공격 관측 가능성 (관찰 결과)}
\begin{figure}[H]
  \centering
  \includegraphics[width=0.88\linewidth]{ex/topology_attack_observability.png}
  \caption{Topology별 공격 관측 가능성(PDR drop)과 공격 경유 노출률}
  \label{fig:obs}
\end{figure}

\begin{table}[H]
\centering
\caption{Topology별 관측 지표(attack rate 50\% 선행 실험 로그 기반, 우연 관찰)}
\label{tab:obs}
\begin{tabular}{lrrrr}
\toprule
Topology & Normal PDR (\%) & Attack PDR (\%) & PDR Drop (\%p) & Exposure (\%) \\
\midrule
Topo-1 & 100.00 & 47.83 & 52.17 & 60 \\
Topo-2 & 80.00 & 39.13 & 40.87 & 60 \\
Topo-3 & 84.06 & 49.28 & 34.78 & 100 \\
Topo-4 & 80.00 & 47.83 & 32.17 & 60 \\
Topo-5 & 60.00 & 30.43 & 29.57 & 0 \\
Topo-6 & 60.00 & 60.00 & 0.00 & 0 \\
\bottomrule
\end{tabular}
\end{table}

\noindent
관찰 포인트: (i) 동일 공격률에서도 PDR Drop이 0.00\%p에서 52.17\%p까지 크게 변한다.
(ii) 노출률이 0\%인 Topology(Topo-6)는 관측 저하가 거의 없다.
이 결과는 본 연구를 위한 의도된 설계가 아니라, 선행 실험의 부산물로 관찰된 사례임에도 불구하고
``공격률이 같아도 토폴로지/parent 구조에 따라 관측 가능성이 크게 달라진다"는 가설을 떠올리게 한 출발점이다.

\subsection{Parent 조합 관점의 구조 해석}
\begin{figure}[H]
  \centering
  \includegraphics[width=0.88\linewidth]{ex/topology_parent_composition.png}
  \caption{공격 시 최종 parent class 조합(송신 노드 4--8)}
  \label{fig:parent}
\end{figure}

\begin{table}[H]
\centering
\caption{Parent class 비율 요약(\%)}
\label{tab:parent}
\begin{tabular}{lrrrrr}
\toprule
Topology & Via attacker & Via relay & Direct root & No parent & Other \\
\midrule
Topo-1 & 0 & 40 & 0 & 0 & 60 \\
Topo-2 & 100 & 0 & 0 & 0 & 0 \\
Topo-3 & 60 & 0 & 0 & 0 & 40 \\
Topo-4 & 60 & 0 & 0 & 0 & 40 \\
Topo-5 & 0 & 40 & 0 & 0 & 60 \\
\bottomrule
\end{tabular}
\end{table}

\section{시스템/위협 모델 및 관측 가능성 정의}
본 절에서는 실험에서 가정하는 네트워크/공격/관측 모델을 요약하고,
관측 가능성(Detectability)의 정량적 정의를 제시한다.

\subsection{시스템 및 위협 모델(개요)}
상세한 네트워크/트래픽/공격자 모델은 연구계획서 본문에서 확장 기술할 예정이며, 여기서는 핵심 가정만 요약한다.
\begin{itemize}
  \item 네트워크: RPL/BRPL 기반 IoT/LLN, 단일 루트, 고정 노드 배치
  \item 트래픽: 주기적 UDP 송신, 동일 패킷 크기, 일정/랜덤 주기 혼합
  \item 공격자: 선택된 중간 노드가 수신 패킷의 일부를 드롭(Selective Forwarding), 라우팅 프로토콜 자체는 준수
  \item 관측자: 루트 및 로그 수집기가 PDR/Delay를 측정, 링크 품질 변동과 MAC 충돌 등 노이즈 존재
\end{itemize}

\subsection{관측 가능성(Detectability) 정의(개요)}
공격률을 $\alpha$라 하고, 노드 $i$의 공격 경유 확률을 $P_{\text{attack}}(i)$라 할 때,
이론적으로는 $\alpha\,P_{\text{attack}}(i)$가 충분히 클수록 PDR Drop이 통계적으로 유의하게 나타날 것으로 예상한다.
관측 가능성은 효과 크기(예: $\Delta\text{PDR} > \delta$)와 통계적 분리(신뢰구간 비중첩 등)를 통해 정의하며,
실제 분석에서는 윈도우 기반 단기/장기 관측 등으로 확장할 수 있다.

\begin{figure}[H]
  \centering
  % 산점도/회귀선 가설 그림 (PDF 벡터 그림 사용)
  \includegraphics[width=0.8\linewidth]{figures/fig_hypothesis_exposure_vs_pdr}
  \caption{공격 경유 확률과 PDR Drop 간 가설적 관계(개념도, synthetic 데이터 예시)}
  \label{fig:hypo-exposure-pdr}
\end{figure}

\noindent
본 절에서는 시스템/위협 모델의 개요와 관측 가능성의 정성적 정의를 제시하였다.

\section{연구 설계 (본 실험 계획)}
\subsection{공통 설정(초안)}
\begin{itemize}
  \item 시뮬레이터: Cooja (Contiki-NG)
  \item 노드 수: 20/40/60 (규모 민감도)
  \item 트래픽: 주기적 UDP (예: 30초 간격)
  \item 시뮬레이션 시간: 30--60분
  \item 반복 횟수: 시나리오당 10--30회
  \item 라우팅: RPL vs BRPL
  \item 공격률: $\alpha \in \{0,0.2,0.4,0.6,0.8,1.0\}$
\end{itemize}

\subsection{시나리오 매트릭스}
\begin{table}[H]
\centering
\caption{토폴로지/구조 시나리오 설계}
\label{tab:scenario}
\begin{tabular}{p{2.6cm}p{3.2cm}p{4.2cm}p{3.2cm}}
\toprule
시나리오 & 설계 의도 & 기대 구조 지표 특징 & 공격 노드 위치 \\
\midrule
Star-like & Root 주변 집약 & APL 낮음, PD 낮음 & leaf 또는 주변부 \\
Chain/Tree-like & 중계 의존 강조 & APL 높음, PD 낮음 & 필수 중계 노드 \\
Mesh-like & 경로 다양성 강조 & APL 중간, PD 높음 & 중간 노드 \\
BRPL Adaptive & 동적 parent 전환 & PD 증가, BC 분산 & 중심 또는 경계 \\
\bottomrule
\end{tabular}
\end{table}

\begin{figure}[H]
  \centering
  % 시나리오 매트릭스 (PDF 벡터 그림 사용)
  \includegraphics[width=0.9\linewidth]{figures/fig_scenario_matrix}
  \caption{토폴로지/프로토콜/공격률 조합에 대한 시나리오 매트릭스(설계 개요도)}
  \label{fig:scenario-matrix}
\end{figure}

\subsection{분석 절차}
\begin{enumerate}
  \item 시나리오별 실험 수행 및 로그 수집
  \item 구조 지표(APL, PD, BC, $P_{\text{attack}}$) 추출
  \item 관측 지표(PDR, E2E delay)와 상관/회귀 분석
  \item RPL vs BRPL 비교 및 통계적 유의성 검정
\end{enumerate}

\section{분석 방법 및 통계 계획(개요)}
실제 연구에서는 구조 지표와 관측 지표 간 상관/회귀 분석, RPL vs BRPL 비교, 혼합효과모형 등 통계 기법을 사용하여
관측 가능 조건의 수량화를 시도할 예정이다.
본 연구계획서 초안 수준에서는 다음과 같은 분석 흐름을 상정한다.
\begin{itemize}
  \item 구조 지표(APL, PD, BC, $P_{\text{attack}}$)와 PDR Drop/Delay 변화 간 상관 분석
  \item 공격률, 토폴로지 시나리오, BRPL 여부 등을 공변량으로 포함한 회귀 또는 혼합효과모형 초안 수립
  \item RPL vs BRPL의 관측 가능성 차이를 효과크기와 신뢰구간 중심으로 비교
\end{itemize}

\section{기대 기여}
\begin{itemize}
  \item 공격 탐지 \emph{기법}이 아닌 공격 \emph{관측 가능 조건} 정량화
  \item 동적 라우팅 환경에 맞춘 구조 지표 기반 분석 프레임 제시
  \item 실험 결과와 수학 모델의 대응 관계(설명 가능성) 제공
\end{itemize}

\section{향후 보완 항목}
\begin{itemize}
  \item 통계 검정 구체화(ANOVA/혼합효과모형/신뢰구간)
  \item 시간 창(Window)에 따른 단기/장기 관측 분리 분석
  \item 공격 노드 위치(depth, centrality) 실험 설계 세분화
  \item 관련연구 인용 및 참고문헌 정리
\end{itemize}

\vspace{1em}
\noindent\textbf{한 줄 요약:} 토폴로지는 분류 대상이 아니라 수치로 요약되는 독립 변수다.

\end{document}
