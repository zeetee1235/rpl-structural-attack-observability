\documentclass[10pt,journal,onecolumn]{IEEEtran}

% ---- Korean ----
\usepackage{kotex}
\usepackage[T1]{fontenc}

% ---- Figures/Tables ----
\usepackage{graphicx}
\graphicspath{{figures/}{tables/}}
\usepackage{booktabs}
\usepackage{amsmath,amssymb}
\usepackage{csvsimple}
\usepackage{enumitem}

% ---- Hyperref & URL handling ----
\usepackage[hidelinks]{hyperref}
\usepackage{xurl}
\usepackage{url}
\urlstyle{same}
\emergencystretch=2em

% ---- Float control ----
\usepackage[section]{placeins}

% ---- Float spacing ----
\setlength{\textfloatsep}{10pt plus 2pt minus 2pt}
\setlength{\floatsep}{8pt plus 2pt minus 2pt}
\setlength{\intextsep}{8pt plus 2pt minus 2pt}

% ---- Optional: compact lists ----
\setlist[itemize]{noitemsep, topsep=2pt}
\setlist[enumerate]{noitemsep, topsep=2pt}

\title{On the Observability of Selective Forwarding Attacks in RPL/BRPL-based IoT Networks}

\author{이건형}

\begin{document}
\maketitle

\begin{abstract}
본 문서는 RPL/BRPL 기반 IoT 네트워크에서 선택적 포워딩(Selective Forwarding) 공격이
\emph{언제 성능 지표로 관측 가능한지}를 분석한다.
기존 연구들이 공격 탐지 기법 자체에 집중한 반면,
본 연구는 공격률 증가가 항상 PDR 또는 지연 저하로 이어지지 않는다는
실험적 관찰에서 출발한다.
우리는 공격 노드를 실제로 경유하는 트래픽의 정도를 \emph{구조적 노출(Exposure)}로 정의하고,
Contiki-NG 및 Cooja(headless) 시뮬레이션으로
토폴로지 구조와 경로 다양성이 관측 가능성에 미치는 영향을 정량적으로 분석한다.
예비 실험(총 61회 실행) 결과,
공격률보다 노출이 관측 가능성을 지배하는 핵심 요인임을 확인하였다.
또한 본 문서는 재현 가능한 실험 워크플로우(시나리오 정의--실행--로그 수집--분석--리포트)와
RPL-Classic 기반 BRPL 구현 아키텍처를 함께 제시하여, 향후 대규모 실험 확장 기반을 마련한다.
\end{abstract}

\begin{IEEEkeywords}
RPL, BRPL (Backpressure RPL), Selective Forwarding Attack, Observability, Structural Exposure, Low-Power and Lossy Networks
\end{IEEEkeywords}

% ===================== 1. Introduction =====================
\section{서론}
선택적 포워딩 공격은 IoT/WSN 환경에서 대표적인 내부 공격이다.
공격 노드는 라우팅 제어는 정상 수행하되 데이터 패킷 일부를 선택적으로 폐기하여
전달률(PDR) 저하 및 지연 증가를 유발한다.
그러나 실험 환경에서 반복적으로 관찰되는 현상은 다음과 같다.
공격률 $\alpha$를 증가시켜도 PDR/지연 변화가 미미하거나,
토폴로지에 따라 효과가 과도하게 달라지는 사례가 발생한다.
즉, ``공격률 증가 $\Rightarrow$ 관측 가능한 성능 저하''라는 단순 가정은 항상 성립하지 않는다.

본 연구는 \emph{관측 가능성(observability)} 관점에서 문제를 재정의한다.
공격이 존재하더라도, 공격 노드가 실제 트래픽 경로에 포함되지 않거나
경로 다양성(path diversity) 및 parent 전환으로 우회가 발생하면
성능 지표 변화가 희석되어 공격이 보이지 않을 수 있다.
따라서 공격률 자체보다 ``트래픽이 공격자를 얼마나 경유하는가''가
관측 가능성을 좌우할 가능성이 높다.

이를 위해 본 문서는 공격 노드 경유 정도를 \emph{구조적 노출(Exposure)}로 정의하고,
다양한 구조 시나리오(A/B/C/D)에서 공격률을 변화시키며
PDR 중심의 관측 결과를 비교한다.

\subsection{연구 질문}
본 문서는 다음 질문에 답하고자 한다.
\begin{itemize}
  \item 동일 공격률에서, 왜 어떤 구조에서는 공격이 명확히 관측되고 어떤 구조에서는 관측되지 않는가?
  \item 노출(Exposure)과 경로 다양성(Path Diversity)이 관측 지표(PDR/지연)와 맺는 관계는 무엇인가?
  \item 네트워크 규모(노드 수 증가)는 관측 가능성에 어떤 영향을 주는가?
\end{itemize}

\subsection{기여(중간보고 범위)}
본 문서는 탐지 알고리즘을 제안하지 않는다.
대신 다음을 기여로 제시한다.
\begin{itemize}
  \item (C1) 선택적 포워딩 공격의 관측 가능성을 노출 개념으로 정식화하고, 예비 실험으로 1차 검증한다.
  \item (C2) headless Cooja 기반 재현 가능한 실험 파이프라인과 구조화 로그 포맷을 제시한다.
  \item (C3) RPL-Classic 기반 BRPL 구현 아키텍처(Queue--DIO 확장--QuickTheta/Beta--parent 선택)를 정리하여 후속 실험 확장의 기반을 마련한다.
\end{itemize}

% ===================== 2. Background =====================
\section{배경}
\subsection{RPL 개요}
RPL은 LLN 환경에서 IPv6 라우팅을 제공하는 프로토콜이며,
DODAG(Directed Acyclic Graph) 구조를 형성한다.
노드는 루트까지의 경로 비용을 랭크(rank)로 나타내며,
DIO/DAO 제어 메시지에 기반해 parent를 선택한다~\cite{rfc6550}.
목적 함수로는 OF0~\cite{rfc6552}와 MRHOF~\cite{rfc6719}가 대표적이다.

\subsection{BRPL 개요}
BRPL(Backpressure RPL)은 RPL의 경로 비용과
백프레셔(backpressure) 정보를 결합해
트래픽/혼잡 상황에서 라우팅을 적응적으로 조정하는 확장이다~\cite{tahir2018brpl}.
본 연구는 RPL-Classic 기반으로 BRPL 구성요소(큐 상태, DIO 큐 옵션, 가중치 기반 parent 선택)를 구현하고,
선택적 포워딩 공격 관측 가능성 실험에 활용한다.

% ===================== 3. System & Threat Model =====================
\section{시스템 및 위협 모델}
\subsection{네트워크 모델}
본 연구는 단일 루트, 정적 배치(static placement) LLN을 고려한다.
각 송신 노드는 루트로 주기적 UDP 트래픽을 생성한다.
라우팅은 RPL Classic 또는 BRPL로 동작하며, 동일한 시나리오 구성에서 토글된다.

\subsection{공격 모델}
공격자는 내부 노드로 가정하며,
자신을 경유하는 데이터 패킷을 확률 $\alpha\in[0,1]$로 폐기한다.
라우팅 제어(DIO/DAO)는 정상 수행하여,
공격이 제어-plane에서 노출되지 않도록 한다.
즉, 공격은 data-plane에 국한된다.

\subsection{관측 가능성 정의}
공격이 \emph{관측 가능}하다는 것은,
정상 상태 대비 PDR 또는 지연이 통계적으로 유의미한 수준으로 변화함을 의미한다.
본 중간보고에서는 우선 PDR 중심으로 관측 가능성을 정리하며,
95\% 신뢰구간(Confidence Interval) 비중첩, 또는 명확한 효과 크기(예: PDR 감소)가 확인되는 경우를 관측 가능으로 간주한다.
이는 탐지 알고리즘 자체가 아닌 관측 가능성을 다루기 위한 운영상 정의(operational definition)이며,
본 보고서는 공격 탐지 기법 제안이 아니라 이러한 관측 가능성의 1차 정리에 초점을 둔다.

% ===================== 4. Experimental Methodology =====================
\section{실험 방법론}
본 절은 ``시나리오 정의--실행--로그--분석--리포트''로 이어지는
실험 워크플로우를 재현 가능 관점에서 정리한다.

\subsection{실험 워크플로우(자동화 파이프라인)}
Fig.~\ref{fig:workflow}는 전체 워크플로우를 요약한다.
시나리오(\texttt{.csc}) 정의 후 펌웨어를 빌드하고,
headless Cooja로 반복 실행한다.
각 실행(run)은 \texttt{COOJA.testlog}를 산출하며,
분석 스크립트가 이를 파싱해 요약 CSV 및 그림/표를 생성한다~\cite{cooja2006,contiking}.

\begin{figure}[htbp]
  \centering
  \includegraphics[width=0.88\linewidth]{fig1_workflow.pdf}
  \caption{실험 워크플로우 개요: 시나리오 정의 $\rightarrow$ 펌웨어 빌드 $\rightarrow$ headless Cooja 실행 $\rightarrow$ 로그 수집 $\rightarrow$ 분석/리포트}
  \label{fig:workflow}
\end{figure}

\subsection{구성요소 및 코드 구조}
레포 구성은 아래와 같이 역할이 명확히 분리되어 있다.
\begin{itemize}
  \item 시나리오: \path{simulations/scenarios/*.csc}
  \item 펌웨어: \path{simulations/firmware/rpl-node.c} (UDP 트래픽/공격자 모드/OBS 로그)
  \item 실행기: \path{scripts/run_cooja_headless.py} (headless 실행 및 환경변수 주입)
  \item 실행 래퍼: \path{scripts/run_experiments.sh}
  \item 분석기: \texttt{scripts/analyze\_results.py} (PDR*, drop rate, 95\% CI 계산)
\end{itemize}

\subsection{시나리오 설계(A/B/C/D)}
구조적 특성을 제어하기 위해 네 가지 시나리오를 구성하였다.
Fig.~\ref{fig:topologies}는 시나리오 배치를 요약한다.
\begin{itemize}
  \item A (Low Exposure): 공격 노드가 경로에 포함되기 어려운 구조
  \item B (High Exposure): 공격 노드가 필수 중계가 되는 구조 (B(10), B(20))
  \item C (High Path Diversity): 경로 다양성이 높은 구조
  \item D (중심성 변화): 평균 경로 길이는 유사하되 중심성 차이를 유도한 구조
\end{itemize}

\begin{figure}[htbp]
  \centering
  \includegraphics[width=0.92\linewidth]{fig2_topologies.pdf}
  \caption{시나리오 A/B/C/D 토폴로지 요약(루트/공격자/노드 배치).}
  \label{fig:topologies}
\end{figure}

\subsection{노출(Exposure) 지표 정의 및 근사}
노출은 ``공격 노드를 실제로 경유한 트래픽의 비율''로 정의한다.
이 값은 공격률과 독립적으로 토폴로지/parent 선택 구조에 의해 결정될 수 있다.
본 중간보고에서는 구현 및 측정 가능성을 고려하여,
run별 로그에서 관측된 경유 비율(요약 CSV의 \texttt{exposure\_e1\_prime})을 노출의 근사치로 사용한다.
Fig.~\ref{fig:exposure}에서 노출과 PDR 변화의 관계를 시각화한다.

\subsection{평가지표 및 통계 요약}
\textbf{PDR 정의 및 보정.}
일부 로그에서 PDR이 1을 초과하는 현상이 관찰되었다.
이는 TX/RX 카운트 기준 불일치 또는 집계 오차 가능성이 있으므로,
본 보고서는 $PDR^*=\min(PDR,1.0)$로 클리핑하여 보고한다.

\textbf{신뢰구간.}
시나리오$\times\alpha$ 조합별 반복 실행 결과로 평균 및 95\% CI를 산출하여 표로 제시한다.

\subsection{실험 파라미터}
Table~\ref{tab:params}는 실험 파라미터를 요약한다.
(테이블 파일은 \texttt{tables/table1\_sim\_params.csv}에서 로드한다.)

\begin{table}[htbp]
  \centering
  \caption{실험 파라미터(요약)}
  \label{tab:params}
  {\catcode`\_=12
  \csvautotabular{tables_table1_sim_params.csv}}
\end{table}
\FloatBarrier

% ===================== X. Structural Exposure Modeling =====================
\section{Structural Exposure Modeling}
본 절은 선택적 포워딩 공격의 \emph{관측 가능성}을 설명하기 위한
핵심 구조 변수로서 \emph{구조적 노출(Structural Exposure)}을 정의하고,
RPL의 parent 선택(및 switching) 통계를 이용해 노출을 추정하는 모델을 제시한다.

\subsection{Ground-truth Exposure 정의}
공격 노드 $a$, 루트 $r$, 송신 노드 집합 $S$를 고려한다.
관측 구간 $T$ 동안 생성된 모든 데이터 패킷 집합을 $\mathcal{P}_T$라 하자.
각 패킷 $p \in \mathcal{P}_T$의 실제 전달 경로(중계 노드 시퀀스)를 $\mathrm{Path}(p)$라 할 때,
\textbf{Ground-truth Exposure}는 다음과 같이 정의한다.
\begin{equation}
E_T(a) \triangleq \frac{|\{p \in \mathcal{P}_T : a \in \mathrm{Path}(p)\}|}{|\mathcal{P}_T|}.
\label{eq:exposure_gt}
\end{equation}
즉, 전체 송신 트래픽 중 공격 노드를 실제로 경유한 트래픽의 비율이다.
이는 공격률 $\alpha$와 독립이며(드랍 확률이 아닌 경유 확률),
토폴로지 및 라우팅(parent 선택)에 의해 결정된다.

\subsection{정적 수렴 구간에서의 트리 기반 Exposure}
RPL이 수렴하여 관측 구간 동안 parent가 고정된다고 가정하면,
라우팅 구조는 루트로 향하는 트리(또는 유사 트리)로 근사 가능하다.
이때 공격 노드 $a$의 자손(서브트리) 집합을 $\mathrm{Desc}(a)$라 하면,
각 송신 노드 $i \in S$의 트래픽 생성률을 $\lambda_i$로 두었을 때,
트리 기반 노출은 다음과 같이 표현된다.
\begin{equation}
E_{\mathrm{tree}}(a) \triangleq
\frac{\sum_{i \in S} \lambda_i \cdot \mathbf{1}[i \in \mathrm{Desc}(a)]}{\sum_{i \in S} \lambda_i}.
\label{eq:exposure_tree}
\end{equation}
이는 ``공격자 서브트리에 속한 송신자들의 트래픽 비중''에 해당한다.

\subsection{Parent switching을 반영한 시간평균 Exposure}
현실적으로는 시간에 따라 preferred parent가 변할 수 있다(parent switching).
이를 반영하기 위해, 노드 $i$가 이웃 후보 $j \in \mathcal{N}_i$를 parent로 선택하는
시간 비율(또는 확률)을 $\pi_{i \to j}$로 정의한다.
이때 ``노드 $i$에서 생성된 패킷이 최종적으로 공격자 $a$를 경유할 확률''을 $q_i$로 두면,
다음의 재귀식(흡수 상태 포함)으로 정의할 수 있다.
\begin{align}
q_i &= \sum_{j \in \mathcal{N}_i} \pi_{i \to j} \, q_j, \quad \forall i \notin \{a,r\} \label{eq:q_recursion}\\
q_a &= 1, \qquad q_r = 0. \label{eq:q_boundary}
\end{align}
그러면 시간평균 노출은 다음과 같이 계산된다.
\begin{equation}
E_{\mathrm{mix}}(a) \triangleq
\frac{\sum_{i \in S} \lambda_i \, q_i}{\sum_{i \in S} \lambda_i}.
\label{eq:exposure_mix}
\end{equation}
직관적으로 $E_{\mathrm{mix}}$는 parent switching 하에서의 경유 확률을 시간평균화한 노출이며,
$\pi_{i \to j}$는 로그로부터 ``parent가 $j$였던 시간 비율''로 추정할 수 있다.

실험에서는 BRPL/RPL 노드가 출력하는 parent 선택/변경 로그(예: \texttt{PARENT} 이벤트)와
구조화된 요약 CSV에서의 parent 구성 비율을 기반으로 $\pi_{i \to j}$를 추정하였다.
구체적으로, 관측 구간 동안 노드 $i$가 parent로 $j$를 보고한 샘플 수를
노드 $i$의 전체 parent 샘플 수로 나누어 row-normalized 확률로 사용한다.
시뮬레이터 로그에는 연속적인 시간 정보가 제한적으로 주어지므로,
본 보고서는 이벤트 발생 횟수 비율을 시간 비율의 근사치로 간주한다.
또한 실제 계산에서는 DODAG가 수렴한 이후의 구간에서만 $\pi_{i \to j}$를 추정하고,
필요 시 각 행이 1이 되도록 row-normalization 및 고립 노드에 대한 연결성 보정을 수행함으로써
식~(\ref{eq:q_recursion})--(\ref{eq:q_boundary})가 항상 해를 갖도록 한다.

\subsection{링크 품질을 포함한 Effective Exposure(선택)}
무선 링크 손실을 반영하기 위해, 링크 $i \to j$의 데이터 전달 성공 확률을 $s_{i \to j}$로 두면,
공격자 경유 확률 재귀식을 다음처럼 확장할 수 있다.
\begin{equation}
q_i = \sum_{j \in \mathcal{N}_i} \pi_{i \to j} \, s_{i \to j} \, q_j.
\label{eq:q_with_link}
\end{equation}
이는 ``공격자를 경유할 경로를 선택했더라도 링크 손실로 인해 공격자에 도달하지 못하는'' 경우를 반영한다.
(본 중간보고에서는 $s_{i \to j}=1$ 근사로 시작하고, 후속 연구에서 ETX 기반 추정 등을 포함한다.)

\subsection{Exposure와 공격률의 분리 및 PDR 근사}
선택적 포워딩 공격에서 공격률 $\alpha$는 ``공격자가 자신을 통과한 패킷을 드랍할 확률''이다.
따라서 공격에 의한 기대 손실은 노출과 공격률의 곱으로 근사할 수 있다.
배경 손실을 무시하면,
\begin{equation}
\mathbb{E}[\mathrm{PDR}] \approx 1 - \alpha \cdot E_{\mathrm{mix}}(a).
\label{eq:pdr_simple}
\end{equation}
배경 손실률을 $L_0$로 두면 다음과 같이 확장 가능하다.
\begin{equation}
\mathbb{E}[\mathrm{PDR}] \approx (1 - L_0)\cdot (1 - \alpha \cdot E_{\mathrm{mix}}(a)).
\label{eq:pdr_background}
\end{equation}
즉, 공격률이 커도 노출이 충분히 작으면(구조적으로 우회되면)
PDR 저하가 거의 관측되지 않을 수 있다.

% ===================== 5. BRPL Implementation =====================
\section{BRPL 구현 아키텍처(RPL-Classic 기반)}
본 절은 BRPL 구현이 실험 결과에 대한 신뢰성을 제공하도록,
모듈 구성과 데이터 흐름을 구조적으로 기술한다.

\subsection{설계 선택: RPL-Classic 기반}
BRPL은 parent 선택과 메트릭 계산을 변경하는 확장이므로,
개입 지점이 명확한 RPL-Classic 기반이 구현/분석에 유리하다.
특히 본 연구는 parent 구성 비율 및 경유 경로를 관측해야 하므로,
라우팅 의사결정 과정의 로깅 가능성이 중요한 요구사항이다.

\subsection{모듈 구성}
BRPL은 \texttt{contiki-ng-brpl}의 RPL-Classic 스택에 다음을 추가/확장한다.
\begin{itemize}
  \item \textbf{Queue Manager}:
  \texttt{contiki-ng-brpl/os/net/routing/rpl-classic/brpl-queue.c}
  \item \textbf{DIO Queue Option}:
  \texttt{contiki-ng-brpl/os/net/routing/rpl-classic/rpl-icmp6.c}
  \item \textbf{BRPL OF/Rank Facade}:
  \texttt{contiki-ng-brpl/os/net/routing/rpl-classic/rpl-brpl.c}
  \item \textbf{QuickTheta/QuickBeta}:
  \texttt{contiki-ng-brpl/os/net/routing/rpl-classic/rpl-brpl.c}
  \item \textbf{RPL DAG/Parent 확장}:
  \texttt{contiki-ng-brpl/os/net/routing/rpl-classic/rpl.h}
\end{itemize}

Fig.~\ref{fig:brpl_arch}는 데이터 흐름을 요약한다.

\begin{figure}[htbp]
  \centering
  \includegraphics[width=0.92\linewidth]{fig3_brpl_architecture.pdf}
  \caption{BRPL 구현 아키텍처: Queue 상태 $\rightarrow$ DIO 큐 옵션 전파 $\rightarrow$ 이웃 큐 상태 갱신 $\rightarrow$ QuickTheta/Beta $\rightarrow$ weight 계산 $\rightarrow$ parent 선택}
  \label{fig:brpl_arch}
\end{figure}

\subsection{Queue Manager 및 DIO Queue Option}
각 노드는 per-DAG 큐 상태를 유지하며,
스케줄링은 LIFO, 큐 full 시 신규 패킷 drop 정책을 사용한다.
큐 길이는 DIO에 옵션 형태로 포함되어 이웃에게 전파된다.
구현 설정은 다음과 같다.
\begin{itemize}
  \item Queue max: 200 packets
  \item Queue scheduling: LIFO
  \item Queue full: drop new packet
  \item DIO queue option: code 0xCE, payload 4 bytes, big-endian
  \item DIO interval: 512--1024ms
  \item MAC/RDC: CSMA / NULLRDC
\end{itemize}

\subsection{Parent 선택(가중치) 및 직관}
부모 선택은 RPL 비용과 백프레셔 항을 결합한 weight를 최소화한다.
\begin{equation}
w_{x,y} = \theta \cdot \hat{p}_{x,y} - (1-\theta)\cdot \widehat{\Delta Q}_{x,y}
\end{equation}
여기서 $\hat{p}_{x,y}$는 정규화된 경로 비용(예: ETX 기반), $\widehat{\Delta Q}_{x,y}$는 정규화된 큐 차등(backpressure) 항이다.
QuickTheta/QuickBeta는 트래픽 및 이웃 변화에 따라 $\theta$를 업데이트한다.
Fig.~\ref{fig:parent}는 $\theta$ 변화에 따른 선택 경향을 개념적으로 보여준다.

\begin{figure}[htbp]
  \centering
  \includegraphics[width=0.88\linewidth]{fig4_parent_selection_model.pdf}
  \caption{Parent 선택 개념도: $\theta$가 클수록 RPL 비용 중심, 작을수록 백프레셔 중심으로 선택}
  \label{fig:parent}
\end{figure}

% ===================== 6. Experimental Results =====================
\section{예비 실험 결과}
본 절은 2026-02-04 수행한 예비 실험 결과(총 61회 실행)를 요약한다.
실험 매트릭스는 시나리오(A/B/C/D)와 공격률 $\alpha$를 조합하여 구성하였으며,
B(20)에서 노드 수 증가 효과를 함께 확인하였다.

\subsection{실험 실행 요약}
총 61회 실행이 수행되었으며,
\texttt{simulations/output/scenario\_*.log} 및 \texttt{*\_COOJA.testlog} 기반으로 모두 정상 완료를 확인하였다.
반복 수는 시나리오별로 상이하며,
B(10), C(10)은 $\alpha\in\{0,0.4,0.8,1.0\}$에서 각 5회,
A와 D는 $\alpha\in\{0,1.0\}$에서 각 3회,
B(20)은 $\alpha\in\{0,0.8,1.0\}$에서 각 3회 수행되었다.

\subsection{핵심 결과 1: High Exposure(B)에서 PDR$^*$ 급락}
Fig.~\ref{fig:pdr_b}는 시나리오 B에서 $\alpha$ 증가에 따른 PDR$^*$ 변화를 보여준다.
High Exposure 구조에서는 공격률 증가에 따라 PDR$^*$이 급격히 감소하였고,
$\alpha=1.0$에서 B(10)은 약 0.11, B(20)은 약 0.05까지 감소하였다.
이는 공격 노드가 트래픽 경로에 포함되는 비율이 높을 때,
공격이 성능 지표로 명확히 관측됨을 의미한다.

\begin{figure}[htbp]
  \centering
  \includegraphics[width=0.92\linewidth]{fig5_pdr_vs_alpha_b.pdf}
  \caption{Scenario B(High Exposure)에서 공격률 $\alpha$에 따른 PDR$^*$ 변화(B(10) vs B(20)).}
  \label{fig:pdr_b}
\end{figure}

\subsection{핵심 결과 2: Low Exposure(A/D)에서 공격이 ``보이지 않음''}
Fig.~\ref{fig:pdr_abcd}는 시나리오별 PDR$^*$ 변화를 비교한다.
A와 D는 $\alpha$가 1.0에 가까워져도 PDR$^*$이 거의 1.0을 유지하였다.
이는 공격이 존재하더라도 구조적으로 우회(bypass)될 경우
성능 저하가 관측되지 않을 수 있음을 보여준다.

\subsection{핵심 결과 3: High Path Diversity(C)의 완화 효과}
시나리오 C는 $\alpha$ 증가에 따라 PDR$^*$이 감소하지만,
시나리오 B보다 완만한 감소를 보였다(Fig.~\ref{fig:pdr_abcd}).
이는 경로 다양성이 공격 효과를 희석시키거나 우회 경로를 제공하여
관측 가능성을 낮추는 방향으로 작용할 수 있음을 시사한다.

\begin{figure}[htbp]
  \centering
  \includegraphics[width=0.92\linewidth]{fig6_pdr_vs_alpha_abcd.pdf}
  \caption{시나리오 A/B/C/D에서 공격률 $\alpha$에 따른 PDR$^*$ 변화 비교.}
  \label{fig:pdr_abcd}
\end{figure}

\subsection{규모 효과: B(10) vs B(20)}
노드 수 증가가 관측 가능성을 어떻게 변화시키는지 확인하기 위해
B(10)과 B(20)을 비교하였다.
Fig.~\ref{fig:scale}에서 보이듯,
노드 수 증가 시 PDR 저하가 더욱 커지며 관측 가능성이 강화되는 경향이 확인되었다.
이는 규모 증가가 노출 확률을 증가시키거나,
병목 경로를 강화하여 공격 효과가 더 명확히 드러날 수 있음을 시사한다.

\begin{figure}[htbp]
  \centering
  \includegraphics[width=0.88\linewidth]{fig8_scale_effect.pdf}
  \caption{규모 효과: Scenario B에서 10노드 대비 20노드에서 관측 저하가 더 큼.}
  \label{fig:scale}
\end{figure}

\subsection{노출(Exposure)과 관측 가능성의 관계}
Fig.~\ref{fig:exposure}는 노출과 PDR$^*$ 간 관계를 시각화한다.
노출이 증가할수록 PDR 저하가 커지는 경향이 관찰되었으며,
이는 ``공격률보다 노출이 관측 가능성을 지배''한다는 중심 가설을 1차적으로 지지한다.
본 단계에서는 parent switching 통계로부터 계산된 $E_{\mathrm{mix}}$를 사용하였다.

\begin{figure}[htbp]
  \centering
  \includegraphics[width=0.88\linewidth]{fig7_exposure_vs_pdr.pdf}
  \caption{노출($E_{\mathrm{mix}}$)과 PDR$^*$의 관계.}
  \label{fig:exposure}
\end{figure}
\FloatBarrier

\subsection{노출 추정치 비교 및 PDR 근사 검증}
Fig.~\ref{fig:emix_elog}는 로그 기반 노출(E$_{\log}$)과
모델 기반 노출(E$_{\mathrm{mix}}$)의 관계를 비교한다.
또한 Fig.~\ref{fig:pdr_alpha_emix}는 $PDR^*$와 $\alpha\cdot E_{\mathrm{mix}}$의
선형 근사 관계를 시각화한다.

\begin{figure}[htbp]
  \centering
  \includegraphics[width=0.88\linewidth]{fig11_emix_vs_elog.pdf}
  \caption{$E_{\mathrm{mix}}$와 $E_{\log}$ 비교 산점도.}
  \label{fig:emix_elog}
\end{figure}

\begin{figure}[htbp]
  \centering
  \includegraphics[width=0.88\linewidth]{fig12_pdr_vs_alpha_emix.pdf}
  \caption{$PDR^*$ vs. $\alpha \cdot E_{\mathrm{mix}}$ (근사 검증).}
  \label{fig:pdr_alpha_emix}
\end{figure}
\FloatBarrier

\subsection{요약 히트맵 및 parent 구성 분석}
Fig.~\ref{fig:heatmap}는 시나리오$\times\alpha$ 전반의 결과를 요약한다.
또한 Fig.~\ref{fig:parent_comp}는 parent 구성 비율을 통해
Low Exposure 환경에서 공격자를 우회하는 경로가 지배적임을 확인한다.

\begin{figure}[htbp]
  \centering
  \includegraphics[width=0.92\linewidth]{fig10_observability_heatmap.pdf}
  \caption{관측 가능성 요약 히트맵(PDR$^*$).}
  \label{fig:heatmap}
\end{figure}

\begin{figure}[htbp]
  \centering
  \includegraphics[width=0.88\linewidth]{fig9_parent_composition.pdf}
  \caption{Parent 구성 비율: Direct / via attacker / via relay 분해.}
  \label{fig:parent_comp}
\end{figure}

\subsection{정량 요약표}
Table~\ref{tab:pdr_ci}는 시나리오$\times\alpha$ 조합별 평균 PDR$^*$과 95\% CI를 제시한다.

\begin{table}[htbp]
  \centering
  \caption{시나리오$\times\alpha$별 평균 PDR$^*$ 및 95\% CI}
  \label{tab:pdr_ci}
  \footnotesize
  {\catcode`\_=12
  \csvautotabular{tables_table2_pdr_ci.csv}}
\end{table}

% ===================== 7. Discussion =====================
\section{논의}
\subsection{노출이 관측 가능성을 지배하는 이유(구조적 해석)}
예비 실험 결과는 다음의 구조적 해석과 일관된다.
High Exposure 구조(B)에서는 공격 노드가 사실상 cut-vertex/병목 역할을 하며,
다수 송신 트래픽이 공격자를 경유하게 된다.
따라서 $\alpha$ 증가가 곧 PDR 감소로 연결되어 관측이 용이하다.
반면 Low Exposure 구조(A/D)에서는 루트 직접 연결 또는 대체 중계가 지배적이며,
공격 노드가 경로에서 배제되기 쉬워 공격 효과가 숨겨진다.
High Path Diversity(C)에서는 경로 다양성이 우회/분산을 제공하여
공격 효과가 완만하게 나타난다.

\subsection{탐지가 실패하는 구조적 원인(왜 ``보이지'' 않는가)}
관측이 어려운 경우는 단순히 공격이 약해서가 아니라,
\emph{구조적으로 공격 효과가 측정 지표에 반영되지 않는 조건}이 존재하기 때문이다.
대표적인 원인은 다음과 같다.
\begin{itemize}
  \item parent switching: 시간에 따라 parent가 바뀌며 공격 경유가 단속적으로 발생
  \item bypass paths: 루트 직접 경로 또는 대체 중계로 공격자를 회피
  \item path diversity: 다중 경로로 트래픽이 분산되어 단일 공격자의 영향이 희석
\end{itemize}
따라서 탐지 기법 설계 이전에,
해당 네트워크가 구조적으로 관측 가능한지(노출이 충분한지)를 점검하는 것이 중요하다.

\subsection{중간보고 시점의 함의}
본 중간보고 수준에서의 결론은 다음과 같이 요약된다.
\begin{itemize}
  \item 동일 공격률에서도 관측 가능성은 구조(노출)에 의해 크게 달라진다.
  \item 공격이 ``안 보이는'' 경우는 존재하며, 이는 탐지 실패 가능성을 내재한다.
  \item 경로 다양성은 공격 효과를 완화하는 방향으로 작용할 수 있다.
\end{itemize}

% ===================== 8. Threats to Validity =====================
\section{타당도 위협 및 한계(Threats to Validity)}
본 절은 결과 해석의 한계와 향후 개선 방향을 명시한다.

\subsection{측정 타당도(Measurement Validity)}
PDR이 1을 초과하는 현상이 관찰되었다.
이는 RX/TX 카운트 집계 기준의 불일치 가능성이 있으며,
본 보고서는 $PDR^*$로 클리핑하여 보고하였다.
향후에는 TX/RX 정의를 명확히 분리하고,
로그 포인트를 일관된 계층(애플리케이션/네트워크)에서 수집하도록 개선한다.

\subsection{내적 타당도(Internal Validity)}
일부 조합에서 반복 실험의 분산이 0으로 계산되었다.
이는 동일 seed, 결정적 경로 선택, 트래픽 지터 부족 등으로 인해
반복 간 변동성이 충분히 발생하지 않았을 가능성이 있다.
향후에는 seed 분리, 트래픽 지터 강화, 더 긴 실행 시간 및 반복 수 증가로
통계적 신뢰도를 개선한다.

\subsection{외적 타당도(External Validity)}
본 예비 실험은 정적 토폴로지 및 제한된 노드 수에 기반한다.
실제 환경의 동적 링크 품질 변화, 이동성, 간섭, TSCH 등은 고려하지 않았다.
따라서 본 결론은 ``정적 LLN + RPL/BRPL + 특정 시나리오'' 범위에서의 1차 검증으로 해석해야 한다.

\subsection{구성 타당도(Construct Validity)}
노출(Exposure)은 로그 기반 근사치(E1' proxy)로 측정하였다.
후속 연구에서는 노출을 독립 변수로 설계/제어하거나,
parent 선택 확률 모델과 결합하여 보다 엄밀한 지표로 확장한다.

% ===================== 9. Future Work =====================
\section{향후 연구 계획}
중간보고 이후의 구체적 확장 계획은 다음과 같다.
\begin{enumerate}
  \item \textbf{노출/다양성/중심성의 수치 스윕}: 토폴로지를 라벨이 아닌 수치 독립 변수로 생성(\texttt{.csc} 자동 생성 포함)
  \item \textbf{지연/오버헤드 지표 포함}: PDR뿐 아니라 지연, 컨트롤 메시지 오버헤드까지 확장
  \item \textbf{seed/node/지터/반복 수 확장}: 분산 0 이슈 해결 및 신뢰구간 안정화
  \item \textbf{RPL vs BRPL 비교 강화}: 혼잡 조건에서 BRPL의 parent 전환/경유 변화가 관측 가능성에 미치는 영향 분석
\end{enumerate}

% ===================== 10. Conclusion =====================
\section{결론}
본 문서는 RPL/BRPL 기반 IoT 네트워크에서
선택적 포워딩 공격의 관측 가능성이 공격률 자체보다
\emph{구조적 노출(Exposure)}에 의해 지배됨을 예비 실험으로 확인하였다.
High Exposure 구조에서는 공격률 증가에 따라 PDR$^*$이 급락하여 관측이 용이하였고,
Low Exposure 구조에서는 공격이 거의 관측되지 않았다.
High Path Diversity 구조에서는 감소가 완만하여 완화 효과가 관찰되었다.
또한 본 문서는 재현 가능한 실험 워크플로우와 BRPL 구현 아키텍처를 함께 제시하여,
후속 대규모 실험 확장을 위한 기반을 마련하였다.

% ===================== Appendix =====================
\appendices
\section{재현을 위한 산출물 및 경로 요약}
본 보고서의 실험 및 산출물은 다음 경로 체계를 따른다.
\begin{itemize}
  \item GitHub 리포지터리: \url{https://github.com/zeetee1235/rpl-structural-attack-observability}
  \item 시나리오: \path{simulations/scenarios/*.csc}
  \item 펌웨어: \path{simulations/firmware/rpl-node.c}, \path{simulations/firmware/Makefile}
  \item 실행: \path{scripts/run_experiments.sh}, \path{scripts/run_cooja_headless.py}
  \item 로그: \path{simulations/output/*_COOJA.testlog}, \path{simulations/output/scenario_*.log}
  \item 분석: \path{scripts/analyze_results.py}
  \item 요약 CSV: \path{simulations/output/simulation_summary_20260204_230503.csv}
  \item Figure/Table 생성: \texttt{Rscript scripts/generate\_figures.R}
\end{itemize}

\section{Figure/Table 삽입 위치 가이드(요약)}
본 문서에서 사용한 Figure/Table의 목적 및 삽입 위치는 다음과 같다.
\begin{itemize}
  \item Fig.1: 워크플로우(Section 4 서두)
  \item Fig.2: 시나리오 토폴로지(Section 4.3)
  \item Fig.3: BRPL 아키텍처(Section 5 서두)
  \item Fig.4: parent 선택 개념(Section 5.4)
  \item Fig.5/6/7/8/9/10: 결과 및 해석(Section 6--7)
  \item Table 1/2: 파라미터 및 PDR 요약(Section 4, Section 6)
\end{itemize}

\bibliographystyle{IEEEtran}
\bibliography{references}

\end{document}
